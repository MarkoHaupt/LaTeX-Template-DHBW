\section{Analyse der bestehenden Lösung}

	\subsection{Historie}
	
		Bis zu dem Jahr 2016 haben Kunden, die die Ausführung von Batchverfahren anfordern möchten, dies per E-Mail oder telefonisch bei \ac{IE1} getan. Diese ursprüngliche Vorgehensweise hatte einige Nachteile:
		
		\begin{itemize}
			\item fehlende oder Mangelhafte Kommunikation der Laufanforderungen
			\item kein standardisiertes Format für Laufanforderungen 
			\item keine zentralisierte Bearbeitungsstelle
			\item großer manueller Aufwand
		\end{itemize}

		Um den Anforderungsprozess dieser Batchverfahren zu verbessern, wurden im Jahre 2016 Eingabemasken auf HTML-Basis erstellt. Durch diese konnte der Prozess der Anforderung von Batchverfahren bei \ac{IE1} generalisiert und zentralisiert werden. Das Ziel der folgenden Ausführungen ist die Beschreibung der Funktionsweise und die Erörterung der Stärken und Schwächen dieser im Jahre 2016 eingeführten Lösung.      

	\subsection{Prozessmodellierung der bestehenden Lösung}
	
		Wenn Fachbereiche Batchverfahren anfordern möchten, nutzen sie hierzu Eingabemasken auf HTML-Basis. Zum aktuellen Zeitpunkt benötigen die Fachbereiche hauptsächlich die Möglichkeit drei verschiedene Batchverfahren anzufordern, Bestandskorrekturen, Massenpolicierung und AB01/ABUJ24 Nachmeldungen. Der fachliche Inhalt dieser Batchverfahren ist im Rahmen dieser Projektarbeit nicht relevant.
		
		Sobald ein Kunde eines dieser drei Batchverfahren anfordern möchte, öffnet er hierzu die jeweilige HTML-Eingabemaske über seinen Datei-Explorer. Diese Masken bieten dem Kunden die Möglichkeit alle nötigen Laufanforderungen für die Batchverfahren anzugeben. Abbildungen dieser drei Eingabemasken sind im Anhang zu finden. \myFootRef{sec:AnhangZurIstAufnahme} Nachdem der Kunde auf den \myQuote{Speichern}-Knopf drückt, wird eine Datei mit allen Daten in einem festen Format erstellt und es öffnet sich automatisch ein Mailprogramm mit einem Entwurf einer Mail. In dieser werden sowohl der Empfänger, als auch alle Laufanforderungen übernommen, die der Kunde zuvor in der Maske angegeben hat. Der Nutzer muss diese Mail anschließend noch verschicken. 
		
		Daraufhin geht die Mail im Gruppenpostfach von \ac{IE1} ein, dort kann ein Mitarbeiter die Kundenanforderung überprüfen und entscheiden, ob er das Batchverfahren mit diesen Laufanforderungen starten kann. Sollte es keine Probleme oder Bedenken geben, startet er das Batchverfahren manuell über das \ac{AWI}.
		
		\begin{figure}[H]
			\centering
			\includegraphics[width=\linewidth]{resource/Ist_Prozess_V3.png}
			\caption{BPMN-Modell der bestehenden Lösung}
			\label{fig:Ist_Prozess}		
		\end{figure}
	
	\subsection{Evaluation der bestehenden Lösung}
		
		\subsubsection{Vorteile dieser Lösung}
		
			Wie sich im laufe der Nutzung gezeigt hat, hat die beschriebene Lösung einige Vorteile mit sich gebracht.
		
			\paragraph{Zentralisierung}
				
				Dadurch, dass das Gruppenpostfach von \ac{IE1} automatisch als Empfänger festgelegt wird, gehen alle Anfragen an einer zentralen Stelle ein. Dies hat einige Vorteile, zum einen müssen sich neue Kunden nicht nach einem Empfänger erkundigen, bevor sie die Anfrage verschicken können. Ein weiterer Vorteil ist, dass Anfragen nicht mehr an die falschen Mitarbeiter geschickt werden können, die die Anfrage dann wiederum weiterleiten müssten.  
			
			\paragraph{Formbindung}
			
				Durch das Einsetzen einer Eingabemaske ist die Vollständigkeit und Qualität der Nutzereingaben deutlich gestiegen. Es ist nun nur noch in Ausnahmefällen nötig, dass die Mitarbeiter von \ac{IE1} wegen fehlender oder unzulässiger Laufanforderungen bei dem Fachbereich nachfragen muss.  
		
		\subsubsection{Verbesserungsmöglichkeiten}
		
			Diese Lösung hat jedoch auch einige Nachteile die als Verbesserungsmöglichkeiten gesehen werden können.
		
			\paragraph{Automatisiertes Starten des Batchverfahrens}
			
				Der größte Nachteil dieser Lösung ist, dass die Batchverfahren nach Eingang der Anforderung weiterhin manuell von \ac{IE1} gestartet werden müssen. Durch eine Automatisierung des eigentlichen Startens der Batchverfahren könnte \ac{IE1} entlastet werden.     
			
			\paragraph{Simplifizierung des verwendeten Tech-Stacks}

				Ein weiter Nachteil ist, dass die Übermittlung der Anfragen weiterhin per E-Mail stattfindet und die Eingabemasken auf HTML-Basis neue eigenständige Komponenten sind. Hierdurch ist die Komplexität des Prozesses gestiegen und es sind mehr mögliche Fehlerquellen aufgekommen. Durch die Nutzung von HTML, JavaScript und ActiveX wurde die Anzahl der verwendeten Sprachen ebenfalls erhöht, hierdurch könnte die Wartung und Instandhaltung in Zukunft aufwendiger werden.  
				
			\paragraph{Rechtemanagement}
			
				Die Kunden haben mit dieser Lösung die Möglichkeit unabhängig von ihrer Abteilung alle drei verfügbaren Batchverfahren anzufordern. Unbefugte Anforderungen können zwar nach dem Eingang bei \ac{IE1} abgelehnt werden, jedoch sollte bereits das Anfordern an sich bei unzureichenden Berechtigungen vermieden werden, um manuellen Aufwand und mögliche Fehlerquellen weiter zu reduzieren. 
		