\section*{Abstract}
\addcontentsline{toc}{section}{Abstract}

	\begingroup
	\begin{table}[h!]
		\setlength\tabcolsep{0pt}
		\begin{tabular}{p{3.7cm}p{11.7cm}}
			Titel & Portierung und Optimierung des Anforderungsprozesses für das Starten von Batchverfahren - Self-Service für das zentrale Scheduling\\
			Verfasser: & Marko Haupt \\
			Kurs: & WWI21SEA \\
			Ausbildungsstätte: & SV Informatik GmbH\\
		\end{tabular}
	\end{table}
	\endgroup
	
	Self-Service Angebote haben in den letzten Jahren immer mehr an Bedeutung gewonnen und ein Ende dieses Trends ist nicht in Sicht. Durch Customer Self-Service Angebote bekommen Kunden die Möglichkeit ihre Anliegen selbst zu lösen. Die rasanten Entwicklungen in diesem Bereich haben gezeigt, dass Kunden diese Möglichkeit häufig bevorzugen. Die Vorteile enden jedoch nicht bei den Kunden, auch Unternehmen profitieren von einem umfangreichen Self-Service Angebot. Ob zur Steigerung der Kosteneffizienz oder der Erhöhung der Kundenzufriedenheit, Self-Service Angebote sind in der heutigen Zeit nicht mehr wegzudenken.
	
	Kunden der SV Informatik GmbH müssen in sporadischen Abständen das Starten von Batchverfahren anfordern. Hierfür wurde im Jahre 2016 eine Lösung implementiert, die den Kunden bei diesem Prozess unterstützt. Durch eine Änderung der unternehmensweiten Sicherheitsrichtlinien kann diese Lösung jedoch nicht weiter eingesetzt werden und es ist eine Portierung notwendig.
	
	Das Ziel der vorliegenden Projektarbeit ist daher die Portierung und Optimierung dieses Anforderungsprozesses. Durch die Verwendung der Produktsuite von \acf{AA} soll diese Möglichkeit des Self-Service für die Kunden der SV Informatik GmbH weiter bestehen bleiben und verbessert werden. Daraus ergibt sich die Frage, welche Vor- und Nachteile bietet die Portierung des Anforderungsprozesses für das Starten von Batchverfahren auf die Automic Automation-Produktsuite durch Verwendung des \acf{DSRP}?	
	        
    Zur Umsetzung dieses Ziels wurde den Prinzipien des \acf{DSRP} gefolgt. Als Teil der Analyse und Entwicklungsphase wird die bestehende Lösung ausgiebig analysiert, anschließend wird ein Konzept für die neue Lösung erstellt und es erfolgt die Praktische Umsetzung dieses Konzepts. 
    
    Wie aus der Evaluation der neuen Lösung hervorgehen wird, konnte der Anforderungsprozess erfolgreich portiert und verbessert werden.