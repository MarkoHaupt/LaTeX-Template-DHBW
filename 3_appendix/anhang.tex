\section{Anhang}

	\subsection{Anhang zur Analyse der bestehenden Lösung}
	\label{sec:AnhangZurIstAufnahme}
	
		\begin{figure}[H]
			\centering
			\includegraphics[scale=0.8]{resource/Ist_Bestandskorrekturen.png}
			\caption{Ist-Zustand der Eingabemaske Bestandskorrekturen}
			\label{fig:Ist_Bestandskorrekturen}		
		\end{figure}
	
		\begin{figure}[H]
			\centering
			\includegraphics[scale=0.8]{resource/Ist_Massenpolicierung.png}
			\caption{Ist-Zustand der Eingabemaske Massenpolicierung}
			\label{fig:Ist_Massenpolicierung}		
		\end{figure}
	
		\begin{figure}[H]
			\centering
			\includegraphics[scale=0.8]{resource/Ist_AB01-ABUJ24-Nachmeldung.png}
			\caption{Ist-Zustand der Eingabemaske AB01/ABUJ24 Nachmeldung}
			\label{fig:Ist_AB01-ABUJ24-Nachmeldung}		
		\end{figure}			
	
	\newpage
	\subsection{Anhang zur Praktischen Umsetzung der neuen Lösung}
	
		\begin{figure}[H]
			\centering
			\includegraphics[scale=0.7]{resource/VARAIE1.png}
			\caption{VARA-Objekt für IE1}
			\label{fig:VARAIE1}		
		\end{figure}
		
		\begin{figure}[H]
			\centering
			\includegraphics[width=\linewidth]{resource/Scripts.png}
			\caption{Im Form Designer hinterlegte Scripts}
			\label{fig:Scripts}		
		\end{figure}
	
		\paragraph{Erklärung zu Abbildung \ref{fig:Scripts}: \nameref{fig:Scripts}}
		
			Die ersten beiden Funktionen lesen sich den Client und die Systembezeichnung aus dem Target-Scheduler Feld aus dem System Tab der Form-Definition aus.\myFootRef{sec:SystemTab} Die untere Funktion liest das aktuelle Jahr aus und subtrahiert anschließend 1, um das vorherige Jahr zu erhalten.
	
		\begin{figure}[H]
			\centering
			\includegraphics[width=\linewidth]{resource/Neu_Bestandskorrekturen.png}
			\caption{Neue Eingabemaske Bestandskorrekturen}
			\label{fig:Neu_Bestandskorrekturen}		
		\end{figure}
		
		\begin{figure}[H]
			\centering
			\includegraphics[width=\linewidth]{resource//Neu_Massenpolicierung.png}
			\caption{Neue Eingabemaske Massenpolicierung}
			\label{fig:Neu_Massenpolicierung}		
		\end{figure}
	
		\begin{figure}[H]
			\centering
			\includegraphics[width=\linewidth]{resource/JOBP_AB01ABUJ24.png}
			\caption{Jobnetz von AB01/ABUJ24}
			\label{fig:JOBP_AB01ABUJ24}		
		\end{figure}
		
		\begin{figure}[H]
			\centering
			\includegraphics[width=\linewidth]{resource/JOBP_Bestandskorrekturen.png}
			\caption{Jobnetz von Bestandskorrekturen}
			\label{fig:JOBP_Bestandskorrekturen}		
		\end{figure}	
		
		\begin{figure}[H]
			\centering
			\includegraphics[width=\linewidth]{resource/JOBP_Massenpolicierung.png}
			\caption{Jobnetz von Massenpolicierung}
			\label{fig:JOBP_Massenpolicierung}		
		\end{figure}
