\documentclass[paper=a4, fontsize=12pt, toc=bibliography]{scrartcl}

%Standart
\usepackage[utf8]{inputenc}								%Zeichenkodierung, u.u. für Umlaute 
\usepackage[ngerman]{babel}								%Silbentrennung
\usepackage[T1]{fontenc}								%
\usepackage{color}										%Text kann farbig dargestellt werden
\usepackage{amssymb}									%Mathematische Symbole
\usepackage{amsthm}										%Mathematische Umgebeungen
\usepackage{graphicx}									%Bilder und PDFs
\usepackage[hidelinks=true]{hyperref}					%Hyperlinks ohne Markierung
\usepackage{booktabs}									%bessere Tabellen
\usepackage[german=quotes]{csquotes} 					% correct quotes using \enquote{}
\usepackage[final]{microtype}							%Microtype (bessere darstellung der Wörter)
\usepackage{todonotes}									%To do Comments
\usepackage{float}										%Figure placement
\usepackage[parfill]{parskip}							%No indent after line break
\usepackage{lipsum}    									%Blindtext
\usepackage{graphicx} 									%Use various graphics formats
\usepackage[german]{varioref} 							%Nicer references \vref
\usepackage{caption}									%better Captions
\usepackage{booktabs} 									%Nicer Tabs
\usepackage{array}										%
\usepackage{listings}									%Code in Text


%ALGORITHMS
\usepackage{algorithm}
\usepackage{algpseudocode}
\renewcommand{\listalgorithmname}{Algorithmenverzeichnis}
\floatname{algorithm}{Algorithmus}

\definecolor{dkgreen}{rgb}{0,0.6,0}
\definecolor{gray}{rgb}{0.5,0.5,0.5}
\definecolor{mauve}{rgb}{0.58,0,0.82}

\lstset{frame=tb,
	language=Java,
	aboveskip=3mm,
	belowskip=3mm,
	showstringspaces=false,
	columns=flexible,
	basicstyle={\small\ttfamily},
	numbers=none,
	numberstyle=\tiny\color{gray},
	keywordstyle=\color{blue},
	commentstyle=\color{dkgreen},
	stringstyle=\color{mauve},
	breaklines=true,
	breakatwhitespace=true,
	tabsize=3
}

%Fuß und Kopfzeilen
\usepackage[headsepline, footsepline]{scrlayer-scrpage}
\pagestyle{scrheadings}
\automark[section]{section}
\clearpairofpagestyles

\setkomafont{pagehead}{\sffamily\small\upshape}
\setkomafont{pagefoot}{\sffamily\footnotesize\upshape}

\ofoot{\pagemark}	
\ifoot{DHBW Mannheim}


%Zitierung
\usepackage[backend=biber, autocite=footnote, style=authoryear, dashed=false]{biblatex}
\setlength{\bibparsep}{\parskip}						%add space between biblatex entries in the bibliography
\addbibresource{literatur/Literatur.bib}				%Add file bibliography.bib as biblatex resource

%Abkürzungsverzeichnis
\usepackage[printonlyused]{acronym}

%Formatierung
\linespread{1.15}										%Zeilenabstand		
\usepackage{helvet}
\renewcommand{\familydefault}{\sfdefault}	
\usepackage{subfigure}
\usepackage[paper=a4paper, left={2.5cm}, right={2.5cm}, top={2.5cm}]{geometry}
%\setcounter{secnumdepth}{5}			
%\setcounter{tocdepth}{5}		
\setcounter{biburllcpenalty}{7000}					%Linebreak in URL
\setcounter{biburlucpenalty}{8000}					%Linebreak in URL

%Stores
\newcommand{\varKindOfDocument}{Projektarbeit}
\newcommand{\varTitle}{!!!Titel der Arbeit!!!}
\newcommand{\varCourseName}{Wirtschaftsinformatik}
\newcommand{\varFieldOfStudy}{Software Engineering}
\newcommand{\varName}{!!!Name!!!}
\newcommand{\varMatriculationNumber}{!!!Matrikelnummer!!!}
\newcommand{\varCompany}{SV Informatik GmbH}
\newcommand{\varDepartment}{!!!Abteilung!!!}
\newcommand{\varCourse}{!!!Kurs!!!}
\newcommand{\varCourseDirector}{Prof. Dr. Thomas Holey}
\newcommand{\varScientificAdvisor}{Prof. Dr. Thomas Holey}
\newcommand{\varScientificAdvisorMail}{thomas.holey@dhbw-mannheim.de}
\newcommand{\varScientificAdvisorNumber}{+49 621 4105-1115}
\newcommand{\varCompanySupervisor}{!!!Firmenbetreuer!!!}
\newcommand{\varCompanySupervisorMail}{!!!Mail von Firmenbetreuer!!!}
\newcommand{\varCompanySupervisorNumber}{!!!Nummer von Firmenbetreuer!!!}
\newcommand{\varTimeframe}{!!!tt.mm.jjjj - tt.mm.jjjj!!!}

%Eigene Commands
\newcommand{\myFrontmatter}{
	\cleardoublepage
	\pagenumbering{Roman}
}
\newcommand{\myMainmatter}{
	\cleardoublepage
	\pagenumbering{arabic}
	\setcounter{page}{1}
	\ihead{\headmark}
}	
\newcommand{\myFootRef}[1]{\footnote{Siehe: \ref{#1} \nameref{#1}}}
\newcommand{\myLineRef}[1]{\ref{#1} \nameref{#1}}

\newcommand{\myQuote}[1]{\glqq #1\grqq}